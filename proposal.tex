\documentclass[11pt, margin=1in]{article}
\usepackage{amsmath}
\usepackage{amsfonts}
\usepackage{amssymb}
\usepackage[margin=1in]{geometry}
\usepackage{fancyhdr}
\usepackage{natbib}

% Use these for theorems, lemmas, proofs, etc.
\newtheorem{theorem}{Theorem}
\newtheorem{lemma}[theorem]{Lemma}
\newtheorem{proposition}[theorem]{Proposition}
\newtheorem{claim}[theorem]{Claim}
\newtheorem{corollary}[theorem]{Corollary}
\newtheorem{definition}[theorem]{Definition}
\newtheorem{fact}[theorem]{Fact}
\usepackage{tikz}
\usetikzlibrary{arrows}
\newenvironment{proof}{\noindent {\it Proof.}}{\hfill\rule{2mm}{2mm}}
\pagestyle{fancy}
\lhead{\textbf{ECON2099: Final Proposal}}
\rhead{\textit{Melissa Yu}}
\cfoot{\thepage}
\renewcommand{\headrulewidth}{0.4pt}
\renewcommand{\footrulewidth}{0.4pt}
\newcommand{\card}[1]{\ensuremath{\left\vert#1\right\vert}}
\newcommand{\diff}[1]{\, d#1}
\newcommand{\eval}[2]{\Big|_{#1}^{#2}}

\makeatletter

\begin{document}
	
\title{Registries to Improve the Volunteer Labor Market}
\author{Melissa Yu}
\date{}
\maketitle

\section{Introduction}

Since the recession, nonprofits' services have been in high demand, for everything from disaster relief to advocacy and education. Volunteers form the backbone of nonprofits and are key to their success; yet, volunteer work is plagued with inefficiencies resulting from a lack of coordination. For one, volunteers are hard to come by, and even supposing that a nonprofit manages to recruit a group of volunteers, it may often be the case that the group doesn't have the right skill set for the task, or that there is a surplus of volunteers during an ``off-season,'' both of which are inefficient uses of valuable volunteer time and good-will. Moreover, the nonprofits which may have the highest demand for volunteers at any given time may also have the fewest resources available to recruit and screen suitable volunteers. The high density of nonprofits (many of which have similar goals) only complicates the task of matching volunteers to organizations, since volunteers must themselves sort through vast repositories of online listings to apply to jobs suited to their skills, a burdensome chore for a casual volunteer. Finally, some nonprofits with similar missions may be better off collaborating on projects with shared volunteers. 

Recently, \cite{blood-registry} has proposed the use of a centralized clearinghouse registry mechanism to address coordination issues such as seasonal shortages found in the altruistic blood market, which shares many properties with the volunteer labor market. In this proposal, we apply the central ideas of the blood registry to volunteer labor and generalize its ideas to volunteer skill matching and networks of multiple nonprofits. We motivate this research by describing how the proposed mechanism can increase supply-side flexibility in the volunteer market, improve disaster response efficiency, and encourage better saturation of volunteers across nonprofits.

\section{Background}

\subsection{Altruistic Supply}
Altruistic markets (such as blood donation and volunteer labor) are characterized by suppliers who are \textit{not} primarily self-interested, and are instead at least partly motivated by outcome-based social preferences such as inequity aversion \cite{volunteer-registry}. However, because suppliers in these markets do not provide their goods and services for monetary compensation, inefficiencies can arise from the lack of a price. Specifically, participants often cannot gauge how much the market values their supply, leading to both over-supply (in which costs are incurred providing a good that is not valued) and under-supply (in which a good is not offered even though it is highly valued). Thus,  research in the field has focused on the development of market designs that can effectively inform and coordinate volunteers to produce a supply that matches aggregate demand, at the lowest cost to suppliers. 

\subsection{Registries for the Blood Market}
Blood donation is one example of an altruistic market characterized by frequent imbalances between supply and demand, ranging from annual winter shortages to excess supply in the wake of high-profile disasters \cite{blood-market}. The works \cite{blood-registry, volunteer-registry} present one coordination solution for this market in the form of a clearinghouse mechanism, termed a \textit{registry}. This central organization encourages supply-side flexibility by leveraging marginal donors who have largely altruistic motives, but need a device to enforce commitment and may experience solicitation dis-utility. The registry accumulates aggregate demand information together with registered suppliers' donation preferences and invites appropriate volunteers to assist if and only if there is excess demand. By signing up with the registry, volunteers join a pool of ``candidates'' who have all indicated their willingness to donate blood during supply shortages. In turn, the registry promises to contact them infrequently. Participants additionally submit conditions under which they prefer to be called upon by the registry (e.g., only for local shortages or only for highly critical situations). When the normal supply is not sufficient to meet demand, candidates in the registry are ranked by their submitted donation preferences, and the most compatible volunteers (i.e., those with the lowest costs when all else is equal) are invited to help. An invitation signals that an unmet demand exists and that other donors have \textit{not} been invited to fill this particular demand; conversely, the lack of an invitation from the registry signals that there is either no demand, or that another supplier has already been called upon to fulfill the demand. This design requires that the registry have knowledge of the optimum number of donors needed for any given shortage.

The registry improves efficiency in two ways. Firstly, it increases the expected benefits of donation by increasing the likelihood that help is actually needed when a member donates and reducing the likelihood that help is needed when a member is not invited to donate. Secondly, the mechanism naturally selects for members with the greatest preference to help in any given situation, with those having the highest costs neither joining nor helping directly and those with intermediate and high willingness to assist joining the registry with the corresponding preferences. Moreover, the registry induces suppliers who would not otherwise regularly participate in the market to contribute during shortages, and thus does not alter the normal flow of the market. 

Participants are able to coordinate with each other and with aggregate demand to achieve an efficient outcome. In both lab \cite{volunteer-registry} and field \cite{blood-registry} experiments, the authors found that compared to control groups, registry members were more likely to volunteer and were more responsive to shortage appeals.

\subsection{The Volunteer Labor Market}
The volunteer labor market suffers from many of the same coordination problems as the blood market, but also presents unique difficulties. 

In the U.S., about a 62 million people volunteer every year. However, the rate of public service has been declining for over a decade; 25.6\% of the population volunteered through an organization at least once in 2013, down from 28.8\% in 2002 \cite{nonprofit-stats}. A large proportion (over one fifth) of the volunteering community can be considered ``marginal,'' clocking between 1 and 14 hours in a year. While many different volunteer-nonprofit matching services are available online, marginal volunteers are unlikely to use these sites and may not otherwise volunteer without some commitment device. At the same time, there has been an increase in demand for volunteer services; between 2002 and 2012, the number of registered nonprofits grew 8.6\%, with public charities growing by 29.6\% \cite{nonprofit-stats}. Given these trends, nonprofits are struggling to recruit an adequate number of volunteers to sustain their missions. The increasingly crowded nonprofit landscape makes it more difficult and confusing for interested persons to explore the available charity options, discouraging exploration of and participation in the market. As a consequence, most volunteers only work with one or two organizations; in 2015, 72\% of volunteers served with just one organization \cite{nonprofit-stats}. The difficulty of exploration also leads to the starvation of smaller, less well-known groups which are crowded out by larger players. In these cases, smaller organizations may benefit by teaming up and leveraging shared volunteers \cite{too-many-nonprofits}. Increasing supply by engaging more marginal volunteers and facilitating easier discovery of smaller organizations and collaboration between groups through shared ``volunteer pools'' have become promising avenues of exploration.

Like blood banks, many nonprofits also experience seasonal volunteer shortages during the summer and post-holiday periods \cite{angel-food, maryland-food}, which can prevent them from operating at full capacity. However, there is also substantial evidence of excess volunteer supply during the holidays and particularly after disasters, making it unlikely that volunteer shortages are due to a lack of altruism. As an example, Project Angel Food reported a fivefold increase in the number of volunteers over their regular numbers on Thanksgiving day, while no volunteers were signed up for the days between Christmas and New Years or January \cite{angel-food}. The large influx of volunteers following widely-publicized natural disasters can also place a burden on local relief operations, as they divert precious local resources from recovery and duplicate the work of other organizations in the field. The National Service publishes a guidebook titled \textit{Managing Spontaneous Volunteers in Times of Disaster} \cite{managing-volunteers} specifically to offer tips to prevent over-supply from leading to what many call the ``second disaster'' after the real disaster. These observations indicate that suppliers in the volunteer labor market are responsive to perceived increases in demand, but that uncoordinated action between volunteers and nonprofits is common. We suggest that both under-supply and over-supply could be eliminated if volunteers were better informed and coordinated.

\section{Research Proposal}
The previously mentioned attributes of the volunteer labor problem motivate us to suggest a registry based solution. We begin by describing a more direct application of the registry to individual nonprofits, and then discuss a ``pooled'' registry structure for groups of nonprofits with cross-registered volunteers.

\subsection{Within Individual Organizations}
In this model, a nonprofit creates a volunteer registry, which contains a collection of candidates who have agreed to be contacted in cases of critical volunteer shortages for the given nonprofit. This reserve pool is sourced from the set of, people who have volunteered with the nonprofit at some point outside of the last $m$ months.

First, consider the case where only general volunteers are needed; that is, no specific skills are demanded during critical shortage requests. This assumption is reasonable for many volunteer opportunities, such as sorting donations in a food bank, running a homeless shelter, or cleaning facilities. For each volunteer $v \in \mathcal{V}$, we collect his/her general volunteering preferences $\rho^v$, which includes acceptable deployment locations, time availability, and other ``observable'' traits which would allow us to reasonably predict the degree to which a volunteer is compatible with a request. Note that $\rho^v$ is the analogue of the donation preferences solicited by the blood registry. However, this information is more important here than in blood donation due to the larger commitment entailed by volunteering; while blood donation takes about one hour, a volunteer shift might take several hours and require more physical or emotional exertion. Moreover, because the range of possible volunteer opportunities is much larger than the number of types of blood shortages that can occur, the problem of calculating candidates' compatibilities with a new opportunity using just their volunteer preferences is more difficult. A shortage $\theta$ is associated with a set of attributes $\phi^\theta$, describing the same properties as $\rho^v$, and the number of ``reserve'' volunteers needed, $\kappa^\theta$. 

For each contract $(v, \theta)$, we compute a similarity metric $c(v, \theta) \triangleq \tilde{c}(\rho^v, \phi^\theta) \in [0, 1]$ (e.g., cosine similarity), which models the probability a candidate accepts a request based only on the observable properties of the match. The compatibility may also include a ``cool-down'' effect for recently contacted volunteers in order to minimize repeated solicitations in a short time frame. We assume that no two similarity metric values are the same. After ranking all volunteers according to this strict preference, the registry can use an invite-once strategy \cite{volunteer-registry} and contact the top $\kappa^\theta$ candidates, taking all who accept.

Now, consider the skilled volunteer problem, in which each candidate additionally has an ability $\nu_s^v \in [0, 1]$ over each service in some fixed skill set $\mathcal{S}$. Although each volunteer can act in multiple capacities, they can be assigned to at most one role. We assume that volunteers have uniform preferences over each skill they are capable of, and thus compatibility can still be solely determined by observable traits. Each shortage also has a service quota, denoted by $\kappa_s^\theta \in \mathbb{Z}_{\geq 0}$. Note that the quantity of services demanded is an integer for ease of specification, but volunteers may have fractional abilities, denoting ``comfortableness'' with the service. While the blood market effectively only has 4 services, one for each type of blood, $\lvert\mathcal{S}\rvert$ is much larger for the labor market, and skill mismatches have a much more negative impact on the success of a project. In order to identify the candidates who should be contacted, we introduce binary variables $\iota(v, s, \theta)$ which take on the value 1 if $v$ has been selected to perform $s$ in the opportunity $\theta$ and 0 otherwise. The optimization problem can be formulated as the following integer program:
\begin{align*}
\max \quad 
& \sum_v \sum_s c(v, \theta) \iota(v, s, \theta) \\
\text{subject to} \quad 
& \sum_v \nu_s^v \iota(v, s, \theta) \leq \kappa_s^\theta \quad \forall s \in \mathcal{S} \\
& \sum_s \iota(v, s, \theta) \leq 1 \quad \forall v \in \mathcal{V} \\
& \iota(v, s, \theta) \in \{0, 1\} \quad \forall v \in \mathcal{V}, s \in \mathcal{S}
\end{align*}
The objective is to maximize the total compatibility of all selected volunteers, subject to constraints that prevent us from exceeding the desired number of volunteers for each service type or giving a volunteer more than one role. This IP is simply the multiple knapsack problem (MKP) with knapsack-specific weights.

\subsection{Between Networks of Organizations}
We now examine an extension of the previous framework to a network of organizations. Practically, this design is appealing because it could allow a group of ``similar'' nonprofits to share a pool of cross-registered volunteers, thus (1) encouraging engagement with a larger selection of organizations, (2) giving less popular nonprofits more exposure and volunteers, and (3) enabling shortages in one area to be filled by excess supply in another area. The registry members are sourced from the set of marginal volunteers in each member organization and additional volunteers who may have attempted to serve at a member when there was already enough supply and been turned down. 

This model can be applied to large organizations (e.g., the Red Cross) which are composed of individual chapters, or community volunteer centers, which facilitate collaboration between and volunteering at local organizations by providing ``master'' listings of opportunities. As far as we can discern through preliminary research, none of these groups currently utilize any volunteer-sharing registry. It is interesting to note that during natural disasters, local volunteer centers are often converted, ad-hoc, into central clearinghouses which coordinate the efforts of multiple nonprofits and provide volunteers with assignments \cite{managing-volunteers}. By building the registry structure into volunteer centers ahead of time, we can significantly improve the efficiency of disaster response systems by cutting out on-the-spot work of adding volunteers and their skills into a system. Because much of the underlying network structure already exists in the previously described cases, implementing the proposed solution would be relatively straightforward and has the potential to significantly improve volunteer market outcomes. 

\subsubsection{Model}
Now, each volunteer may additionally be associated with a reported preference profile $\pi^v$ which ranks the members of the network (e.g., volunteers may prefer the group they worked for in the past over other groups). When calculating the compatibility of a contract $c(v, \theta)$ for some opportunity $\theta \in \Theta$, the ranking of the organization in $\pi^v$ is also taken into account. The output of a request to the registry is the subset of invited volunteers who have agreed to work in the specified opportunities. 

When shortage requests come in at sufficiently spaced intervals, the network registry can operate in a fairly similar fashion to the original single-organization registry. However, when ``simultaneous'' requests (falling within one processing interval) for volunteers enter the registry, the formulation is different. This situation is plausible during seasonal shortages or natural disasters, which may affect several organization members at around the same time. One solution to this problem expands the previously proposed integer optimization problem for skilled volunteers to multiple groups:
\begin{align*}
\max \quad 
& \sum_v \sum_s \sum_\theta c(v, \theta, \pi^v) \iota(v, s, \theta) \\
\text{subject to} \quad 
& \sum_v \nu_s^v \iota(v, s, \theta) \leq \kappa_s^\theta \quad \forall s \in \mathcal{S}, \theta \in \Theta \\
& \sum_s \sum_\theta \iota(v, s, \theta) \leq 1 \quad \forall v \in \mathcal{V} \\
& \iota(v, s, \theta) \in \{0, 1\} \quad \forall v \in \mathcal{V}, s \in \mathcal{S}
\end{align*}
An alternative objective which seeks to maximize the minimum total compatibility across all requests $\Theta$ may be perceived as more ``fair'' to different member organizations and thereby reduce friction within the network:
\[
\max \quad \min_{\theta \in \Theta} \sum_v \sum_s c(v, \theta, \pi^v) \iota(v, s, \theta)
\]

By incorporating $v$'s organization preferences $\pi^v$ directly into the compatibility measure $c$, we naturally add noise (based on the fit of a position) to a volunteer's preferences, preventing starvation of less popular organizations that may end up at the bottom of most preference profiles. An additional concern is removing incentives for organizations to overstate the number of volunteers they need, since the proposed mechanism always supplies them with a number less than or equal to the requested amount. One solution to this may be to add some small amount of random noise to the reported quotas before using them. Another solution is to modify the output of a request to the registry: Instead of placing the burden of contacting candidates on the registry itself, we could instead provide requesting member organizations with the contact information for the full $\kappa^\theta$ volunteers requested and allow members to send invitations themselves.

\section{Conclusion}
This research proposal has proposed an application of blood registries within the volunteer labor market, presented preliminary models to implement the registries, and described several desirable effects of such a mechanism on the volunteer and nonprofit landscape. 


\bibliographystyle{abbrv}
\bibliography{refs}

\end{document}